\section{Determinante}\label{detmenge}

\subsection{Definition \& Eigenschaften}
\begin{itemize}[nosep]
	\item Ist $A$ regulär: $\det(A) \neq 0$\\
	      Ist $A$ singulär: $\det(A) = 0$
	\item Matrix $A$ mit zwei gleichen Spalten/Zeilen, dann $\det(\vec{a}, \vec{b}, \vec{a}) = 0$
	\item Matrix $A$ mit Nullzeile/Nullspalte, dann $\det(A) = 0$
	\item $\det(E) = 1$
	\item $\det(A \cdot B) = \det(A) \cdot \det(B)$
\end{itemize}

\subsection{Gauss}
Summe von allen Pivot-Elemente welche herausgenommen (dividiert) wurden. \textit{Achtung}: Bei Zeilen tauschen Produkt mit (-1) Multiplizieren!
\[
	\begin{vmatrix}
		\circledr{2} & 2 & 4 \\
		4 & 5 & 6 \\
		7 & 8 & 5 \\
	\end{vmatrix}
	\rightarrow
	2 \cdot \begin{vmatrix}
		1 & 1 & 2 \\
		0 & \circledr{1} & -2 \\
		0 & 1 & -9 \\
	\end{vmatrix}
	\rightarrow
	2 \cdot 1 \cdot \begin{vmatrix}
	1 & 1 & 2 \\
	0 & 1 & -2 \\
	0 & 0 & \circledr{-7} \\
\end{vmatrix}
\]

Determinante $2 \cdot 1 \cdot (-7) = -14$

\subsection{Entwicklungssatz}
Mit Hilfe des Entwicklungssatz können Determinanten grosser Matrizen auch von Hand berechnet werden. Dafür sollte eine Zeile oder Spalte mit möglichst vielen Nullen gewählt werden.
\begin{enumerate}[nosep]
	\item Zeile o. Spalte auswählen
	\item Erstes Pivot-Element herausnehmen
	\item Zeile und Spalte des gewählten Pivot abdecken
	\item Elemente mit der Determinante der nicht abgedeckten Elemente multiplizieren
	\item Schritt 2-5 Wiederholen und zum 1. Element addieren/subtrahieren ($\rightarrow$ Vorzeichen-Matrix) \\
	$\begin{array}{|c|c|c|c|}
		\hline
		+ & - & + & - \\ \hline
		- & + & - & + \\ \hline
		+ & - & + & - \\ \hline
		- & + & - & + \\ \hline
	\end{array}$
\end{enumerate}

\noindent \\ \\ \textbf{Beispiel} (Zeile): 
\[
	\begin{vmatrix}
		a & b & c & d \\
		\textcolor{red}{e} & \textcolor{red}{f} & \textcolor{red}{0} & \textcolor{red}{0} \\
		i & j & k & l \\
		m & n & o & p
	\end{vmatrix} = -e
	\begin{vmatrix}
		b & c & d \\
		j & k & l \\
		n & o & p
	\end{vmatrix} +f
	\begin{vmatrix}
		a & c & d \\
		i & k & l \\
		m & o & p
	\end{vmatrix}
\]

\subsection{Spezialfälle}
\subsubsection{2x2 Matrix}
\[
	det\begin{pmatrix}
		a & b \\
		c & d
	\end{pmatrix} = ad - bc
\]

\subsubsection{3x3 Matrix (Sarrus'sche Formel)}
\[
det\begin{pmatrix}
	a & b & c \\
	d & e & f \\
	g & h & i
\end{pmatrix} = aei + bfg + cdh - ceg - afh - bdi
\]
