\section{Matrixzerlegung}
\subsection{LU- und LR-Zerlegung}
Die \textbf{LU}-Zerlegung von $A$ wird durch den Gauss-Algorithmus gefunden. Die entsprechende Vorwärts-Reduktion wird in die $L$ Matrix eingesetzt. Die rechte obere Ecke am ende ist die $U$ Matrix. Kann überprüft werden durch das Multiplizieren von $A = LU$.
\\ \\
Die \textbf{LR}-Zerlegung ist identisch zu LU, jedoch wird hier zusätzlich die Diagonalmatrix $D$ berechnet, um $L'$ und $R$ zu erhalten.
\\ \\
\noindent\textbf{Beispiel}

\[
A = \begin{pmatrix}
	3 & 3 &-3 \\
	-6 & -4 & 2 \\
	3 & 13 & -18
\end{pmatrix}
\]

\noindent LU
{\tiny Die $1$ sind in der $U$ Matrix enthalten.}
\[
\begin{array}{|ccc|}
	\hline
	\textcolor{blue}{3} & 3 &-3 \\
	\textcolor{blue}{-6} & -4 & 2 \\
	\textcolor{blue}{3} & 13 & -18 \\ 
	\hline
\end{array}
\rightarrow
\begin{array}{|ccc|}
	\hline
	1 & 1 &-1 \\
	0 & \textcolor{green}{-2} & -4 \\
	0 &\textcolor{green}{10} & -15 \\
	\hline
\end{array}
\rightarrow
\begin{array}{|ccc|}
	\hline
	1 & 1 &-1 \\
	0 & 1 & -2 \\
	0 & 0 & \textcolor{red}{5}  \\
	\hline
\end{array}
\]
\[
L = \begin{pmatrix}
	\textcolor{blue}{3} & 0 &0 \\
	\textcolor{blue}{-6} & \textcolor{green}{-2} & 0 \\
	\textcolor{blue}{3} & \textcolor{green}{10} &  \textcolor{red}{5}
\end{pmatrix}
\\
U = \begin{pmatrix}
	1 & 1 & -1 \\
	0 & 1 & -2 \\
	0 & 0 & 1
\end{pmatrix}
\] 

\hrule\vspace{0.2cm}

\noindent LR
{\tiny Die $1$ sind in der $L$ Matrix enthalten.}
\[
L \xrightarrow{Diag. Pivots} D = \begin{pmatrix}
	\textcolor{blue}{3} & 0 &0 \\
	0 & \textcolor{green}{-2} & 0 \\
	0 & 0 & \textcolor{red}{5}
\end{pmatrix}
\]

\[
L' = LD^{-1} = \begin{pmatrix}
	1 & 0 & 0 \\ 
	-2 & 1 & 0 \\
	1 & 5 & 1
\end{pmatrix} 
\quad
R = DU = \begin{pmatrix}
	3 & 3 & -3 \\
	0 & 2 & -4 \\
	0 & 0 & 5
\end{pmatrix}
\]

\subsection{Cholesky-Zerlegung}
Nur für symmetrische und positiv definierte Matrizen.

\begin{align*}
	A &= \begin{pmatrix}
		a_{1,1}    & a_{1,2} & a_{1,3} \\
		& a_{2,2} & a_{2,3} \\
		\text{sym} &         & a_{3,3} \\
	\end{pmatrix} = \underbrace{LL^T}_{Zerlegung} \\
	L &= 
	\begin{pmatrix}
		l_{1,1} & 0 & 0 \\
		l_{1,2} & l_{2,2} & 0 \\
		l_{1,3} & l_{2,3} & l_{3,3} \\
	\end{pmatrix}
\end{align*}


\begin{align*}
	l_{1,1} &= \sqrt{a_{1,1}} &\qquad l_{1,2} &= \frac{a_{1,2}}{l_{1,1}} \\
	l_{1,3} &= \frac{a_{1,3}}{l_{1,1}} &\qquad l_{2,2} &= \sqrt{a_{2,2} - l_{1,2}^2} \\
	l_{2,3} &= \frac{a_{2,3} - l_{1,2} \cdot l_{1,3}}{l_{2,2}} &\qquad l_{3,3} &= \sqrt{a_{3,3} - l_{1,3}^2 - l_{2,3}^2} \\
\end{align*}

\subsection{QR-Zerlegung (Gram-Schmidt)}
Die Matrix $Q$ ist orthogonal und die $R$ eine obere Rechtecks-Matrix.
\[A = \begin{pmatrix}
	-1 & 0 & 4 \\
	2 & 4 & -3 \\
	2 & 5 & 2
\end{pmatrix} = Q \cdot R \]

\noindent $A$ orthogonalisieren (Siehe \verweiseref{orthogonalisieren}) und $R = Q^{T}A = Q^{-1}A$ berechnen. Achtung: Alle orthogonalisierten Vektoren in Matrix $Q$ müssen normalisiert sein: $\vec{v}_n = \frac{\vec{v}_n}{|\vec{v}_n|}$.


\subsection{Singulärwertzerlegung}
Die Singulärwertzerlegung zerlegt die Matrix $A$ in $A = U\Sigma V^T$. Wobei $\Sigma$ die Streckungsfaktoren (die Singulärwerte), $U$ die orthogonalisierten Basis-Vektoren und $V$ die ursprünglichen Koordinaten sind.

Beispiel:
\[A = \begin{pmatrix}
	1 & 0 \\
	0 & 1 \\
	1 & 1
\end{pmatrix}
\xrightarrow{\text{Range der Matrix A}}
\Sigma = \begin{pmatrix}
	s_1 & 0 \\
	0 & s_2 \\
	0 & 0
\end{pmatrix}
\]

\noindent Die Matrix $\Sigma$ kann durch das charakteristische Polynom (Siehe \verweiseref{charakteristischempolynom}) mit $\lambda \rightarrow \det(AA^T - \lambda E) = 0$ berechnet werden. \[\lambda = \{\lambda_1 = 0; \lambda_2 = 3; \lambda_3 = 1\}\]

\noindent Die entsprechenden $\lambda$ können der \textit{grösse nach} absteigend sortiert, die $\sqrt{\lambda_n}$ gezogen und als $s_n$ in die Matrix $\Sigma$ eingesetzt werden. Um $U$ und $V$ zu bestimmen müssen $U = \text{eigVc}(AA^T)$ und $V = \text{eigVc}(A^TA)$ gesucht werden (Siehe auch Eigenvektoren). \textbf{Achtung}: entsprechende Eigenvektoren müssen normiert sein!

\[\Sigma = \begin{pmatrix}
	\sqrt{3} & 0 \\
	0 & \sqrt{1} = 1 \\
	0 & 0
\end{pmatrix}
 U = \begin{pmatrix}
	\frac{1}{\sqrt{6}} & \frac{1}{\sqrt{2}} & \frac{1}{\sqrt{3}} \\
	\frac{1}{\sqrt{6}} & -\frac{1}{\sqrt{2}} & \frac{1}{\sqrt{3}} \\
	\frac{2}{\sqrt{6}} & 0 & -\frac{1}{\sqrt{3}} \\
\end{pmatrix}
 V = \begin{pmatrix}
	\frac{1}{\sqrt{2}} & \frac{1}{\sqrt{2}} \\
	\frac{1}{\sqrt{2}} & -\frac{1}{\sqrt{2}}  \\
\end{pmatrix}
\]

\subsection{Pseudoinverse}
Die Pseudoinverse $A^{\dagger}$ kann verwendet werden, um unterbestimmte Gleichungssysteme zu lösen (Überbestimmte können mit \verweiseref{leastsquare} gelöst werden). Die Pseudoinverse hat den Vorteil, dass der gefundene Lösungsvektor immer Minimal ist.


\[\Sigma{\dagger} = \text{Inverse (soweit möglich)} \Sigma^{-1}\]
\[A^{\dagger} = V\Sigma^{\dagger}U^T\]