\section{Eigenwert und Eigenvektor}
Ein Vektor $\vec{v}$ heisst \textbf{Eigenvektor} der quadratischen Matrix $A$ zum \textbf{Eigenwert} $\lambda$, wenn: 
\begin{align*}
	A\vec{v} &= \lambda\vec{v} \Rightarrow (A - \lambda E)\vec{v} = 0  & (\vec{v} &\neq 0)
\end{align*}

\noindent
Ein Eigenvektor $\vec{v}$ wird nur mit der Matrix $A$ gestreckt Es gibt maximal $n$ verschiedene Eigenwerte $\lambda$ jedoch unendlich viele Eigenvektoren $a \cdot \vec{v}$.

\subsection{Eigenwertproblem}\label{charakteristischempolynom}
Es werden $n$ Lösungen für $\lambda$ mit dem \textbf{charakteristischem Polynom} gesucht:
\begin{equation}\label{eq:lambda}
	\det(A - \textcolor{red}{\lambda} E) = 0
\end{equation} Anschliessend kann jedes $\textcolor{red}{\lambda}_n$ in die Formel 
\begin{equation}
	(A - \textcolor{red}{\lambda}_n E)\vec{v}_n = 0
\end{equation} eingesetzt werden um die Eigenvektoren zu bestimmen. \\Die Rechnung kann mit $A\vec{v} \eqq \lambda\vec{v}$ überprüft werden.
\\ \\
\noindent\textbf{Beispiel}
$A = \begin{pmatrix} 0 & 3 \\ -2 & 5 \end{pmatrix}$
\\ \\
\noindent Eigenwert $\lambda_n$ mit (\ref{eq:lambda}) berechnen:
\[\det\begin{pmatrix}
	0 - \lambda & 3 \\
	-2 & 5 - \lambda
\end{pmatrix} \Rightarrow \lambda^2 - 5\lambda + 6 = 0\]
\[{\scriptstyle \lambda_1 = 3; \lambda_2 = 2}\]
\label{eigenvektor}
\noindent Eigenvektor (für $\lambda = 2$) berechnen:
\[
\begin{array}{|cc|c|}
	\hline
	0 - 2 & 3 & 0 \\
	-2 & 5 - 2 & 0 \\
	\hline
\end{array}
\xrightarrow{Gauss}
\begin{array}{|cc|c|}
	\hline
	1 & -\frac{3}{2} & 0 \\
	0 & 0 & 0 \\
	\hline
\end{array}
\]
\[\Rightarrow v_2 = \begin{pmatrix} 3 \\ 2 \end{pmatrix} \quad v_1 = \begin{pmatrix} 1 \\ 1 \end{pmatrix}\]

\subsection{Eigenbasis $A'$}\label{eigenbasis}
Für die Eigenbasis wird $T$ und $T^{-1}$ benötigt. $A' = TAT^{-1}$. $T$ ist die Zusammensetzung von allen Eigenvektoren $v_i$ (Siehe \verweiseref{eigenvektor}).

\subsection{Diagonalisieren}\label{diagonalisieren}
Um Matrix $A$ zu diagonalisieren, muss $A$ nach Matrix $D$ (mit Basen aus Eigenvektoren $\{\vec{v}_1,\dots\vec{v}_n\}$) transformiert werden. Dies ist nur möglich, wenn es eine Basis aus Eigenvektoren gibt. \\
Basis-Transformation $T$ berechnen:
\[T = \begin{pmatrix}
	\vec{v}_1 & \dots & \vec{v}_n
\end{pmatrix}^{-1}\]
Zum Abschluss können die $\lambda$ vom Charakteristischem Polynom eingesetzt und $D$ kontrolliert werden:
\[D = T^{-1}AT = \begin{pmatrix}
	\lambda_1 & \dots & 0 \\
	\vdots & \ddots & \vdots \\
	0 & \dots & \lambda_n \\
\end{pmatrix}\]
